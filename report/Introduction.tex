\chapter{Introduction}
\renewcommand{\baselinestretch}{\mystretch}
\label{chap:Intro}
%\setlength{\parindent}{0pt}

\section{The problem}

\PARstart{V}{ixen} 3 \cite{vixen} is an open-source application for designing and controlling automation lighting displays. With modern technologies, computer-controlled displays are becoming increasingly popular as holiday decorations and other lighting show projects. Vixen is capable of supporting multiple display controllers with thousands of channels for complex lighting shows and effects. It can also synchronise the lighting sequence to audio playback.

By being open-source, Vixen is primarily built towards lower cost Do-It-Yourself solutions. Together with the support of software plugins, users can create customised plugins to support a variety of display controllers, or improve the Vixen software framework itself.

However, the current execution engine used by Vixen has significant computational overheads. Consequently, a powerful computer with sufficient memory space is required to render the lighting effects in real-time. This may not be possible in some situations, increasing the cost of display setup, and is very energy inefficient.

\clearpage

\section{Aims}

\ca{The targets to be achieved are:}

\begin{itemize}[noitemsep]
  \item Develop a new fast playback engine for Vixen application.
  \item Reduce runtime computational requirements.
  \item Control lighting display with tens of thousand of channels from embedded device.
  \item Possibly parallelising some components of the engine.
\end{itemize}

\section{Achievements}

\ca{The achievements of this project are:}

\begin{itemize}[noitemsep]
  \item A new playback engine was developed, integrated into Vixen application.
  \item New export formats from pre-rendering implemented for the new engine.
  \item The new engine reduces layers of overheads, achieved remarkable performance.
  \item A CUI version of Vixen was implemented with portability for embedded platforms.
  \item Because of similarity, sequence input of video format was supported.
  \item Analysed different sets of video formats for an optimal choice between sequence file size and performance.
  \item Stream muxing was implemented for combining video, audio and configuration text, to simplify management and sharing of sequences.
  \item Audio playback was implemented for embedded platforms.
  \item The CUI Vixen application can easily handles thousands of lighting channels on a low-end Raspberry Pi B+, with extendable controller support as modules.
\end{itemize}
