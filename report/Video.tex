\chapter{Video data format}
\renewcommand{\baselinestretch}{\mystretch}
\label{chap:Video}
%\setlength{\parindent}{0pt}

\section{Similarity}

\cmt{The whole process looks similar to animation designing, video rendering and playback}

\section{Implementation}

\cmt{Using ffmpeg for encoding and muxing in export, decoding and demuxing in playback}

\cmt{Using fmod for audio rendering to selectable devices}

\cmt{Using Interop for interfacing C/C++ DLLs to C\#. However, complex C/C++ data structure and processing require additional encapsulation in C/C++ \wn{17}

C/C++ and C\# implementation of basic encoding / decoding for testing.

Audio muxing and metadata for combining files. \wn{18}

\texttt{Makefile} and \texttt{diff} for automatic building and testing. \wn{17}}

\section{Benefits}

\cmt{Uses video to combine sequence dump, audio and configuration information into a single file, with reduced size \wn{16}}

\cmt{Easy sharing, editing software exists, sophisticated compress methods, hardware acceleration possible}

\section{Limitations}

\cmt{Small resolution for large number of channels, may not leverage full advantage of hardware video decoding}

\cmt{Discrete rapid-changing unassociated continuous channels not suitable for common video encoding methods and pixel formats:

Colour space, colour degrade, channel cross-talk}

\section{Integration}

\cmt{C\# interface, integrated directly to \texttt{Vixen.Sys} \wn{18}}

\cmt{Using newer version of fmod\wn{19}}

\cmt{Integrated to playback engine, switch based on file extension}

\section{Performance}

\cmt{Varies encoding, formats, quality factor... Insert data table \wn{16}}
