\chapter{Performance and Analysis}
\renewcommand{\baselinestretch}{\mystretch}
\label{chap:Perf}
%\setlength{\parindent}{0pt}

\section{Platform comparison}

\fref{fig:raw-seq-p-c} clearly shows the relative performance figures of all testing platforms. The NP1380 platform is unable to meet the minimum requirement of 50 fps sequences, therefore most performance analysis for \texttt{VixenConsole} was done on the Raspberry Pi B+ platform.

\section{Implementation comparison}

\begin{figure}[t]
  \centering
  \includegraphics[width=0.8\textwidth]{Figs/RPi-perf.eps}
  \caption{\footnotesize Performances of different implementations on Raspberry Pi B+}
  \label{fig:perf-RPi}
\end{figure}

\fref{fig:perf-RPi} shows the performance comparison between different implementations on Raspberry Pi B+. \texttt{VixenLinky} (referenced by \texttt{raw}) has the highest refresh rate for its simplicity. \texttt{VixenConsole} with the ``Raw'' sequence format (referenced by \texttt{seq}) has the second highest performance. The unlimited playback performance drops to almost half when using a \texttt{rgb24} encoded video as the input (referenced by \texttt{v}). The performance drops again by a small factor when audio stream was also added to the video input (referenced by \texttt{va}). The video only playback performance of \texttt{yuv420p} encoding (referenced by \texttt{v420}) is only a little bit higher than the lossless \texttt{rgb24} encoding (\texttt{v}), whereas the performance of lossless \texttt{yuv444p} encoding (referenced by \texttt{vyuv}) is noticeably lower. Therefore, video encoding format of \texttt{libx264rgb} with \texttt{rgb24} pixel format is more suitable for encoding video sequences.

\cmt{Test different container formats?}

\cmt{Performance of NAS/TX2 to show incomparable performance improvement over original Vixen application}
